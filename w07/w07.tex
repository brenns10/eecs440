\documentclass[fleqn]{homework}

\student{Stephen Brennan (smb196)}
\course{EECS 440}
\assignment{Written 7}
\duedate{October 13, 2015}

\usepackage{mathtools}
%\usepackage{graphicx}

\begin{document}
  \maketitle

  \begin{problem}{1}
    \begin{question}
      A binary classifier is tested on $N$ independent test sets.  The
      classifier makes $r_i$ errors on the $i\textsuperscript{th}$ set, which
      has size $n_i$.  Find the maximum likelihood estimate of the true error
      rate of the classifier. (10 points)
    \end{question}

    For each test set, the sampling distribution is binomial with parameters
    $n_i$ and $e_D$.  We would like to find the estimate of $e_D$ that maximizes
    the likelihood of the observed values $r_i$.  If $N=1$, the maximum
    likelihood estimate of $e_D$ would simply be $\frac{r_i}{n_i}$.  However,
    for $N>1$, we must pick an estimate of $e_D$ that maximizes the likelihood
    of \textit{all} of the observed errors.  Let us refer to each sampling
    distribution as $R_i$.  In mathematical terms, we are looking for the
    following estimate of $e_D$:

    \begin{align}
      e_D &= \arg \max_p Pr\{R_1 = r_1 | n_1, p; R_2 = r_2 | n_2, p; \dots; R_N =
        r_N | n_N, p\} \label{eq:p1-and} \\
      &= \arg \max_p \prod_{i=1}^N Pr\{R_i = r_i | n_i, p\} \label{eq:p1-prod} \\
      &= \arg \max_p \prod_{i=1}^N \binom{n_i}{r_i}
        p^{r_i}(1-p)^{n_i-r_i} \label{eq:p1-binom} \\
      &= \arg \max_p \prod_{i=1}^N p^{r_i}(1-p)^{n_i-r_i}  \\
      &= \arg \max_p \sum_{i=1}^N \ln (p^{r_i} (1-p)^{n_i-r_i}) \label{eq:p1-sum}\\
      &= \arg \max_p \sum_{i=1}^N (r_i) \ln p + (n_i-r_i) \ln(1-p) \label{eq:p1-mid}
    \end{align}

    We can make the step from Equation~\ref{eq:p1-and} to
    Equation~\ref{eq:p1-prod} because the test sets are independent.  We can
    eliminate the binomial coefficient in Equation~\ref{eq:p1-binom} because it
    is constant and won't affect the arg max.  Finally, we can switch to a sum
    of logarithms in Equation~\ref{eq:p1-sum} because the logarithm is an
    increasing function and whatever value of $p$ maximizes the logarithm will
    maximize the original function.

    Next, we let $R_T = \sum_{i=1}^{N} r_i$, and $N_T = \sum_{i=1}^N n_i$.  For
    simplicity of notation, we call the function we are looking to find the
    argmax of (in Equation~\ref{eq:p1-mid}) $z(p)$.  In order to maximize $z$,
    we take the derivative of $z$ and look for critical points:

    \begin{align}
      \frac{dz}{dp} = \frac{d}{dp} \left( R_T \ln p + (N_T-R_T)\ln(1-p) \right)
      &= 0 \\
      \frac{R_T}{p} - \frac{N_T-R_T}{1-p} &= 0 \\
      R_T(1-p) &= p(N_T - R_T) \\
      R_T - R_T p &= N_T p - R_T p \\
      \frac{R_T}{N_T} &= p
    \end{align}

    To ensure that this value of $p$ is the value that maximizes $z$, we must
    perform the second derivative test to ensure that $\frac{d^2z}{dp^2} < 0$.

    \begin{align}
      \frac{d^2z}{dp^2} = -\frac{R_T}{p^2} - \frac{N_T - R_T}{(1-p)^2}
      &\overset{?}{<} 0 \\
      \frac{R_T}{\left(\frac{R_T^2}{N_T^2}\right)} - \frac{N_T -
      R_T}{\left(\frac{(N_T-R_T)^2}{N_T^2}\right)} &\overset{?}{<} 0 \\
      -\frac{N_T^2}{R_T} - \frac{N_T^2}{N_T-R_T} < 0
    \end{align}

    The second derivative test shows that this is indeed a maximum, and
    therefore $e_D = \frac{R_T}{N_T}$ is the maximum likelihood parameter
    estimate.
  \end{problem}

  \begin{problem}{2}
    \begin{question}
      Two classifiers $A$ and $B$ are evaluated on samples of size $n$ and found
      to have error rates $e_A$ and $e_B$ such that $e_A - e_B = 0.1$.  If the
      true error rates of $A$ and $B$ are indeed different, how large does $n$
      have to be to \textbf{guarantee} we can establish the difference at a 95\%
      confidence level? (10 points)
    \end{question}

    In order to establish the difference at a 95\% confidence level, we must
    have $0.1 - 1.96\sigma > 0$:

    \begin{align}
      \sigma &< \frac{0.1}{1.96} \\
      Var(Err_A - Err_B) &< \frac{0.1^2}{1.96^2} \\
      \frac{e_A(1-e_A) + e_B(1-e_B)}{n} &< \frac{0.1^2}{1.96^2} \\ 
    \end{align}

    First, we note that this variance is never negative, so we do not need to
    consider the second case of the inequality caused by the square root.  By
    experience, we know that expression $p(1-p)$ is maximized when $p=0.5$, so
    to guarantee the inequality holds:

    \begin{align}
      \frac{0.5}{n} &< \frac{0.1^2}{1.96^2} \\
      n &> \frac{0.5 * 1.96^2}{0.1^2} \\
      n &> 434.16
    \end{align}
  \end{problem}

  So, a sample of size 435 or greater would guarantee that we can establish the
  difference with 95\%.

  \begin{problem}{3}
    \begin{question}
      Professor X tests a classifier on a test set of size $n$ and determines
      the $C\%$ confidence interval to be $(a,b)$.  Professor Y performs an
      independent test of the same classifier with another test set of size $n$.
      What is the probability that the sample error rate found by Y will lie in
      $(a,b)$? (10 points)
    \end{question}

    When Professor X creates a $C\%$ confidence interval of $(a,b)$, this means
    that the classifier's true error rate will lie within $(a,b)$ with $C\%$
    probability.  This is not the same as saying that the sample error rate on
    an independent sample will lie within $(a,b)$ with that same probability.

    The true parameter values of the sampling distribution of the classifier on
    a sample of size $n$ could be used to give us the probability we're looking
    for.  However, the only estimates of these parameter values are the ones we
    have from Professor X's maximum likelihood estimate.  So, the only answer we
    can give is to use those parameter estimates, giving us a $C\%$ probability
    that the sample error rate found by Y will lie in the confidence interval.
  \end{problem}

  \begin{problem}{4}
    \begin{question}
      A revolutionary new classifier, the Bayesian Deep Kernel Support Neural
      Tree Machine, has been invented.  Professors Bayesian Bob and Neural Nan
      have independently evaluated such a classifier on test sets of size $n$.
      Over dinner, they share their confidence interval findings with each
      other.  Unfortunately, they are overheard by Professor Band Wagon, who
      wants to scoop them without doing the experiment.  What is the best
      confidence interval that Professor Wagon could report that would be
      consistent with Profs. Bob and Nan's findings? (10 points)
    \end{question}

    If Profs. Bob and Nan's confidence intervals overlapped, Professor Wagon
    could give either a $C^2\%$ confidence interval that was the intersection of
    the two intervals, or a $(1-C)^2\%$ confidence interval that was the union
    of the two intervals.  Otherwise, Professor Wagon must give both confidence
    intervals, as he (or she) can't infer a better one.
  \end{problem}

  \begin{problem}{5}
    \begin{question}
      The maximum margin formulation for a linear classifier does not include
      $b$, the constant term, in the objective.  In other words, the choice of
      $b$ plays no role in improving generalizing ability.  Explain intuitively
      why this makes sense. (10 points)
    \end{question}

    When we train a linear classifier, we look for values of the parameters
    $\vec{w}$ and $b$ that satisfy certain constraints.  The constraints are
    that for positive examples $\vec{w}\cdot\vec{x} + b \ge 1$, and for negative
    examples $\vec{w}\cdot\vec{x} + b \le -1$.  We aim to maximize the margin
    (i.e. distance) between these two planes.  When you think of computing the
    distance between the two planes, you think of subtracting, and it
    intuitively makes sense that the value for $b$ would cancel out as you
    determine this distance.  So if you believe that maximizing margin means
    maximizing generalizing ability, it should make sense that $b$ doesn't
    matter for that.
 \end{problem}

\end{document}