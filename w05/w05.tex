\documentclass[fleqn]{homework}

\student{Stephen Brennan (smb196)}
\course{EECS 440}
\assignment{Written 5}
\duedate{September 29, 2015}

\usepackage{mathtools}
%\usepackage{graphicx}

\begin{document}
  \maketitle

  \begin{problem}{1}
    \begin{question}
      Can you think of any circumstances when it might be \textit{beneficial} to
      overfit? (5 points)
    \end{question}

    Yes.  If you encountered a situation where you were certain your training
    data were accurate, and that your training data represents pretty much all
    that your classifier will see later, then overfitting would be good.
    However, in situations like this, machine learning is not usually the
    appropriate approach.  When you have this much knowledge about the target
    concept, you're probably better off trying to use this knowledge directly
    instead of training a classifier with it.  Machine learning is most
    applicable for situations where the target concept is unknown, or too
    complex to directly write it, and in these cases overfitting is not
    desirable.
  \end{problem}

  \begin{problem}{2}
    \begin{question}
      Person $X$ wishes to evaluate the performance of a learning algorithm on a
      set of $n$ examples.  $X$ employs the following strategy: Divide the $n$
      examples randomly into two equal-sized disjoint sets, $A$ and $B$.  Then
      train the algorithm on $A$ and evaluate it on $B$.  Repeat the previous
      two steps for $N$ iterations ($N$ large), then average the $N$ performance
      measures obtained.  Is this sound empirical methodology?  Explain why or
      why not. (10 points)
    \end{question}

    No, at least not if person $X$ then tried to use the mean and standard
    deviation obtained from these experiments to represent the performance of
    the algorithm on the whole population.  When you evaluate the performance of
    a learning algorithm, you do so by attempting to estimate the parameters of
    the error rate distribution on all training sets (of some size) from the
    population.  Since you do not have all training sets, you pretend you do by
    using $k$ fold validation.  This can give you maximum likelihood estimates
    of the parameters of the true performance, and a confidence interval.  When
    you repeat the experiment many times, you are really measuring the
    performance of the algorithm on the various subsets of your training sample,
    and so the mean and variance of the $N$ iterations will not be a reliable
    estimate of the true performance of the algorithm.
  \end{problem}

  \begin{problem}{3}
    \begin{question}
      Two classifiers $A$ and $B$ are evaluated on a sample with $P$ positive
      examples and $N$ negative examples, and their ROC graphs are plotted.  It
      is found that the ROC of $A$ \textit{dominates} that of $B$, i.e. for
      every FP rate, TP rate of $A \ge$ TP rate of $B$.  Discuss what the
      relationship is between the precision-recall graphs of $A$ and $B$ on the
      same sample. (10 points)
    \end{question}

    Since $A$ and $B$ are evaluated on the same sample, we know that the
    statement ``TP rate of $A \ge$ TP rate of $B$'' is equivalent to saying
    $TP_A \ge TP_B$.  Since the ROC graph is monotonically increasing, $A$
    dominating $B$ also means that for every TP rate, FP rate of $A \le$ FP rate
    of $B$, which is the same as $FP_A \le FP_B$.

    Since Recall is plotted on the $x$ axis of the Precision-Recall graph, $A$
    will be to the right of $B$ (since we already have that $TPR_A \ge TPR_B$.
    Precision is defined as $\frac{TP}{TP+FP}$, which can be equivalently
    written as:

    \begin{equation*}
      \frac{1}{1+\frac{FP}{TP}}
    \end{equation*}

    Since $FP_A \le FP_B$ and $TP_A \ge TP_B$, $Precision_A \ge Precision_B$,
    and therefore the Precision-Recall plot of $A$ will be above and to the
    right of $B$ (whenever they differ, that is).
  \end{problem}

  \begin{problem}{4}
    \begin{question}
      Explain why: \textbf{(i)} an ROC graph must be monotonically increasing,
      \textbf{(ii)} the ROC graph of a majority class classifier is a diagonal
      line, \textbf{(iii)} the ROC graph of a random classifier that ignores
      attributes and guesses each class with equal probability is a diagonal
      line. (15 points)
    \end{question}
  \end{problem}

  \begin{problem}{5}
    \begin{question}
      Derive the backpropagation weight updates for hidden-to-output and
      input-to-hidden weights when the loss function is cross entropy with a
      weight decay term.  Cross entropy is defined as
      $L(\vec{w}) = -\sum_i y_i \log \hat{y}_i + (1-y_i)\log(1-\hat{y}_i)$,
      where $y_i$ is true label (assumed 0/1) and $\hat{y}_i$ is the estimated
      label for the $i\textsuperscript{th}$ example. (10 points)
    \end{question}
  \end{problem}

\end{document}