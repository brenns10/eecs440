\documentclass[fleqn]{homework}

\student{Stephen Brennan (smb196)}
\course{EECS 440}
\assignment{Programming 4}
\duedate{November 19, 2015}

%\usepackage{mathtools}
%\usepackage{graphicx}
\usepackage{booktabs}
\usepackage{placeins}

\begin{document}
  \maketitle

  \begin{problem}{a}
    \begin{question}
      For all problems and the two learning algorithms, compare the accuracy of
      the ensemble versions (10 iterations) to the base learners. Produce a
      table with the accuracies of the base learner, the bagged learner, and the
      boosted learner. Perform paired $t$-tests to determine if any of the
      ensemble methods are significantly better than the base learner with 95\%
      confidence.
    \end{question}
    \FloatBarrier

    \begin{table}[h]
    \centering
    \caption{Comparison of base, bagged, and boosted learners.}
    \label{t:a-comparison}
    \begin{tabular}{lllllll}
      \toprule
      Problem & \multicolumn{3}{c}{Decision Tree} & \multicolumn{3}{c}{Neural Network} \\
      \midrule
             & Base          & Bagging       & Boosting      & Base          & Bagging      & Boosting       \\
      Voting & 0.989 (0.007) & 0.982 (0.015) & 0.984 (0.012) & 0.984 (0.012) & 0.986 (0.009) & 0.977 (0.012) \\
      Spam   & 0.711 (0.002) & 0.710 (0.002) & 0.717 (0.005) & 0.647 (0.034) & 0.657 (0.035) & 0.697 (0.003) \\
      Volcanoes&0.672 (0.001)& 0.672 (0.001) & 0.773 (0.018) & 0.833 (0.028) & 0.839 (0.014) & 0.820 (0.023) \\
      \bottomrule
    \end{tabular}
    \end{table}

    A comparison of the accuracy is presented in Table~\ref{t:a-comparison}.
  \end{problem}

  \begin{problem}{b}
    \begin{question}
      For any two problems and the two learning algorithms, evaluate how the
      accuracy of bagging changes with the number of iterations. Pick at least
      three iteration values between 2 and 50, and plot the accuracy on a
      graph. Do you see any difference by problem? By algorithm?
    \end{question}
  \end{problem}

  \begin{problem}{c}
    \begin{question}
      Repeat (b) for boosting.
    \end{question}
  \end{problem}

  \begin{problem}{d}
    \begin{question}
      Evaluate the sensitivity of bagging to noise as follows. When training,
      after constructing the training sample, flip an example’s label with
      probability $p$. Then use this noisy sample in your bagging algorithm and
      evaluate the resulting classifier on the usual (noise free) test set. For
      any two problems and any two learning algorithms, plot a graph with $p$ on
      the $x$-axis and the test-set accuracy of bagging (30 iterations) on the
      $y$-axis. You can use results from the previous questions for a $p=0$
      point. Discuss how resilient bagging is to noise based on your
      observations.
    \end{question}
  \end{problem}

  \begin{problem}{e}
    \begin{question}
      Repeat (d) for boosting.
    \end{question}
  \end{problem}

\end{document}