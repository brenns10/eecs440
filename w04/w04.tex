\documentclass[fleqn]{homework}

\student{Stephen Brennan (smb196)}
\course{EECS 440}
\assignment{Written 4}
\duedate{September 22, 2015}

%\usepackage{mathtools}
%\usepackage{graphicx}

\begin{document}
  \maketitle

  \begin{problem}{1}
    \begin{question}
      Explain in your own words: \textbf{(i)} why memorization should not be
      considered a valid learning approach, \textbf{(ii)} why tabula rasa
      learning is impossible, and \textbf{(iii)} why picking a good example
      representation is important for learning.  Try to use good, intuitive
      examples from human learning to motivate your arguments. (10 points)
    \end{question}
  \end{problem}

  \begin{problem}{2}
    \begin{question}
      Do you think it might be possible to have a ``best'' learning algorithm,
      that would outperform all other algorithms on all learning problems?
      Explain why or why not.  (Hint: think about the consequences of the
      proposition that tabula rasa learning is impossible.) (10 points).
    \end{question}
  \end{problem}

  \begin{problem}{3}
    \begin{question}
      From first principles (without using other results), prove that in a
      binary classification task, the information gain $IG(X)$ for any binary
      split variable $X$ is always non negative. (10 points).
    \end{question}
  \end{problem}

  \begin{problem}{4}
    \begin{question}
      Show that for a continuous attribute $X$, the only split values we need to
      check to determine a split with max $IG(X)$ lie between points with
      different labels.  (Hint: consider the following setting for $X$: there is
      a candidate split point $S$ in the middle of $N$ examples with the same
      label, to the right of $n$ such examples.  To the left, there are $L_0$
      examples with label negative and $L_1$ with label positive, and likewise
      to the right.  Express the information gain of $S$ as a function of $n$.
      Then show that this function is maximized either when $n=0$ or $n=N$.) (20
      points)
    \end{question}
  \end{problem}

\end{document}