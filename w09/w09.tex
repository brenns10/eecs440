\documentclass[fleqn]{homework}

\student{Stephen Brennan (smb196)}
\course{EECS 440}
\assignment{Written 9}
\duedate{November 10, 2015}

\usepackage{enumerate}
%\usepackage{mathtools}
%\usepackage{graphicx}

\begin{document}
  \maketitle

  \begin{problem}{1}
    \begin{question}
      Suppose $K_1$ and $K_2$ are two valid kernels. Show that for positive $a$
      and $b$, the following are also valid kernels: (i) $aK_1+bK_2$ and (ii)
      $aK_1K_2$, where the product is the Hadamard product: if $K=K_1K_2$ then
      $K(x,y)=K_1(x,y)K_2(x,y)$. (10 points)
    \end{question}
  \end{problem}

  \begin{problem}{2}
    \begin{question}
      Define $K(x,y)=(x \cdot y+c)^3$, where $c$ is a constant. Show that $K$ is
      a valid kernel in two ways by (i) finding $\phi$ so that
      $K = \phi(x) \cdot \phi(y)$, and (ii) showing that $K$ is symmetric
      positive semidefinite. (10 points)
    \end{question}
  \end{problem}

  \begin{problem}{3}
    \begin{question}
      Show that, if $K(x,y) = \phi(x) \cdot \phi(y)$ for some $\phi$, it must be
      (a) symmetric and (b) positive semidefinite. (10 points)
    \end{question}
  \end{problem}

  \begin{problem}{4}
    \begin{question}
      An alternative way to think about kernels is as evaluation functions in a
      \textit{“reproducing kernel Hilbert space”} (RKHS), so that finding a
      hyperplane $w$ is equivalent to finding a function $f_w$ in the RKHS of a
      kernel so that $f_w(x)=<w, \phi(x)>$ (the angle brackets are the dot
      product). In this question we will explore this connection. Read the
      ``Brief Introduction to RKHS'' document on the class website. Now suppose
      we have the three points $(x,y)=\{(0,2), (1⁄2,−1), (1,0)\}$ and we want a
      hyperplane in feature space that exactly fits the three points, i.e. we
      want $w$ so that $<w, \phi(x_i)>=y_i$. Suppose the feature space map is
      given by the quadratic kernel $K(a,b)=(ab+1)^2$ so that the feature space
      corresponds to $P_2([0,1])$ as in the document. Answer the following
      questions:

      \begin{enumerate}[(a)]
      \item We can interpret $\phi(x)$ as a representer in $P_2([0,1])$. Which
        polynomials are the representers for $x=0, 1⁄2, 1$? (6 points)
      \item Suppose we restrict $w$ to be a linear combination of representers,
        i.e. $w=\sum α_i \phi(x_i)$. Rewrite the equation $<w, \phi(x_i)>=y_i$
        in terms of the kernel matrix $K$, the vector of $\alpha$’s and the
        vector of $y$’s. (6 points)
      \item Compute the kernel matrix for the three points above and solve your
        answer to (b) for $\alpha_i$.  Given the representers in (a) and the
        values of $\alpha_i$, find the polynomial in $P_2([0,1])$ corresponding
        to $w$. Explain in your own words the connection between this polynomial
        and the hyperplane in feature space. (8 points)
      \end{enumerate}
    \end{question}
  \end{problem}

\end{document}