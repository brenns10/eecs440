\documentclass[fleqn]{homework}

\student{Stephen Brennan (smb196)}
\course{EECS 440}
\assignment{Written 9}
\duedate{November 10, 2015}

\usepackage{enumerate}
\usepackage{mathtools}
\usepackage{xfrac}
%\usepackage{graphicx}

\newcommand{\inner}[1]{\langle #1 \rangle}

\begin{document}
  \maketitle

  \begin{problem}{1}
    \begin{question}
      Suppose $K_1$ and $K_2$ are two valid kernels. Show that for positive $a$
      and $b$, the following are also valid kernels: (i) $aK_1+bK_2$ and (ii)
      $aK_1K_2$, where the product is the Hadamard product: if $K=K_1K_2$ then
      $K(x,y)=K_1(x,y)K_2(x,y)$. (10 points)
    \end{question}

    \begin{enumerate}[(i)]
    \item We have that $\vec{v}^T K_1 \vec{v} \ge 0$,
      $\vec{v}^T K_2 \vec{v} \ge 0$, for any $\vec{v}$.  Now, we compute this
      for $K$:
      \begin{align*}
        \vec{v}^T K \vec{v} \ge 0 
        &= \vec{v}^T (aK_1 + bK_2) \vec{v} \\
        &= a \vec{v}^T K_1 \vec{v} + b \vec{v}^T K_2 \vec{v} \ge 0
      \end{align*}
      Therefore, $K$ is positive semidefinite.  Furthermore, we have
      $K(x,y) = aK_1(x,y) + bK_2(x,y) = aK_1(y,x) + bK_2(y,x) = K(y,x)$, so $K$
      is symmetric.  Therefore, $K$ is a valid kernel.

    \item Since $K_1$ and $K_2$ are valid kernels, we know that there exist
      $\phi_1$ and $\phi_2$ such that $K_1(x,y) = \phi_1(x) \cdot \phi_1(y)$ and
      $K_2(x,y) = \phi_2(x) \cdot \phi_2(y)$.  If we can write $K(x,y)$ in terms
      of some function $\phi'$, then $K$ is also a valid kernel.
      \begin{align*}
        K(x,y) &= aK_1(x,y) K_2(x,y) \\
               &= a(\phi_1(x) \cdot \phi_1(y))(\phi_2(x) \cdot \phi_2(y)) \\
               &= a\left(\sum_{i=1}^n \phi_{1i}(x)\phi_{1i}(y)\right) \left(\sum_{j=1}^m \phi_{2j}(x)\phi_{2j}(y)\right)\\
               &= \sum_{i=1}^n \sum_{j=1}^m a\phi_{1i}(x) \phi_{1i}(y) \phi_{2j}(x) \phi_{2j}(y) \\
               &= \sum_{i=1}^n \sum_{j=1}^m \sqrt{a}\phi_{1i}(x) \phi_{2j}(x) \sqrt{a}\phi_{1i}(y) \phi_{2j}(y) \\
               &= \phi'(x) \cdot \phi'(y)\\
      \end{align*}

      Where $\phi'(x)$ is a vector valued function of length $nm$, and
      $\phi'_{ij}(x) = \sqrt{a} \phi_{1i}(x) \phi_{2j}(x)$.  Therefore, $K$ is
      also a valid kernel.
    \end{enumerate}
  \end{problem}

  \begin{problem}{2}
    \begin{question}
      Define $K(x,y)=(x \cdot y+c)^3$, where $c$ is a constant. Show that $K$ is
      a valid kernel in two ways by (i) finding $\phi$ so that
      $K = \phi(x) \cdot \phi(y)$, and (ii) showing that $K$ is symmetric
      positive semidefinite. (10 points)
    \end{question}

    \begin{enumerate}[(i)]
    \item We start by expanding $K(x,y)$:

      \begin{align*}
        K(x,y) &= (x \cdot y + c)^3 \\
               &= \left(\sum_{i=1}^n x_i y_i + c\right)\left(\sum_{j=1}^n x_j y_j + c\right)\left(\sum_{k=1}^n x_k y_k + c\right)\\
               &= \left(\sum_{i=1}^n \sum_{j=1}^n x_ix_jy_iy_j + \sum_{i=1} 2cx_iy_i + c^2\right)\left(\sum_{k=1}^n x_k y_k + c\right)\\
               &= \sum_{i=1}^n \sum_{j=1}^n \sum_{k=1}^n x_ix_jx_k y_iy_jy_k + \sum_{i=1}^n \sum_{j=1}^n 3cx_ix_jy_iy_j + \sum_{i=1}^n 3c^2x_iy_i + c^3 \\
      \end{align*}

      We let $\phi(x)$ be vector valued with length $n^3 + n^2 + n + 1$, and
      with value:

      \begin{align*}
        \phi(x) = (&x_i x_j x_k, &\dots, &&\:\:\:\: \forall i,j,k \in [1,n] \\
        &\sqrt{3c} x_i x_j, &\dots, &&\:\:\:\: \forall i,j \in [1,n] \\
        &\sqrt{3c^2} x_i, &\dots, &&\:\:\:\: \forall i \in [1,n] \\
        & \sqrt{c^3})\\
      \end{align*}

      Then, we have that $K(x,y) = \phi(x) \cdot \phi(y)$.
    \item First, we show that $K$ is symmetric:
      $K(x,y) = (x \cdot y + c)^3 = (y \cdot x + c)^3 = K(y,x)$, due to the
      property that the dot product is commutative.

      To show that $K$ is semi-definite, we start with the definition:

      \begin{align*}
        \vec{v}^T K \vec{v}=
        &= \sum_{i=1}^n \sum_{j=1}^n v_i v_j K(x_i, x_j) \\
        &= \sum_{i=1}^n \sum_{j=1}^n v_1 v_j (x_i \cdot x_j + c)^3 \\
        &= \sum_{i=1}^n \sum_{j=1}^n v_i v_j \phi(x_i) \cdot \phi(x_j) \\
        &= \sum_{i=1}^n v_i \phi(x_i) \cdot \left(\sum_{j=1}^n v_j \phi(x_j)\right) \\
        &= \left(\sum_{i=1}^n v_i \phi(x_i)\right) \cdot \left(\sum_{j=1}^n v_j \phi(x_j)\right) \ge 0 \\
      \end{align*}

      We use the conclusion from (i), separate the summations due to the
      distributive property of dot products, and then conclude that the overall
      dot product is greater than zero because it is a vector dotted with
      itself, which is always greater than or equal to zero.
    \end{enumerate}
  \end{problem}

  \begin{problem}{3}
    \begin{question}
      Show that, if $K(x,y) = \phi(x) \cdot \phi(y)$ for some $\phi$, it must be
      (a) symmetric and (b) positive semidefinite. (10 points)
    \end{question}

    \begin{enumerate}[(a)]
    \item Since the dot product is commutative, we know
      $K(x,y) = \phi(x) \cdot \phi(y) = \phi(y) \cdot \phi(x) = K(y,x)$, so $K$
      must be symmetric.
    \item We must show that $\vec{v}^T K \vec{v} \ge 0$ for any $\vec{v}$:

      \begin{align*}
        \vec{v}^T K \vec{v} &= \sum_{i=1}^n \sum_{j=1}^n v_i v_j \phi(x_i) \cdot \phi(x_j) \\
        &= \sum_{i=1}^n \left(v_i \phi(x_i) \cdot \sum_{j=1} v_j \phi(x_j) \right) \\
        &= \left(\sum_{i=1}^n v_i \phi(x_i)\right) \cdot \left(\sum_{j=1}^n v_j \phi(x_j)\right) \ge 0\\
      \end{align*}

      Since this is a vector dotted with itself, it must be greater than or
      equal to zero.  (This is the same logic behind my answer to 2(ii)).
    \end{enumerate}
  \end{problem}

  \begin{problem}{4}
    \begin{question}
      An alternative way to think about kernels is as evaluation functions in a
      \textit{“reproducing kernel Hilbert space”} (RKHS), so that finding a
      hyperplane $w$ is equivalent to finding a function $f_w$ in the RKHS of a
      kernel so that $f_w(x)=\inner{w, \phi(x)}$ (the angle brackets are the dot
      product). In this question we will explore this connection. Read the
      ``Brief Introduction to RKHS'' document on the class website. Now suppose
      we have the three points $(x,y)=\{(0,2), (\sfrac{1}{2},−1), (1,0)\}$ and
      we want a hyperplane in feature space that exactly fits the three points,
      i.e. we want $w$ so that $\inner{w, \phi(x_i)}=y_i$. Suppose the feature
      space map is given by the quadratic kernel $K(a,b)=(ab+1)^2$ so that the
      feature space corresponds to $P_2([0,1])$ as in the document. Answer the
      following questions:

      \begin{enumerate}[(a)]
      \item We can interpret $\phi(x)$ as a representer in $P_2([0,1])$. Which
        polynomials are the representers for $x=0, \sfrac{1}{2}, 1$? (6 points)
      \item Suppose we restrict $w$ to be a linear combination of representers,
        i.e. $w=\sum \alpha_i \phi(x_i)$. Rewrite the equation
        $\inner{w, \phi(x_i)}=y_i$ in terms of the kernel matrix $K$, the vector
        of $\alpha$’s and the vector of $y$’s. (6 points)
      \item Compute the kernel matrix for the three points above and solve your
        answer to (b) for $\alpha_i$.  Given the representers in (a) and the
        values of $\alpha_i$, find the polynomial in $P_2([0,1])$ corresponding
        to $w$. Explain in your own words the connection between this polynomial
        and the hyperplane in feature space. (8 points)
      \end{enumerate}
    \end{question}

    \begin{enumerate}[(a)]
    \item The representer $q_x$ is a polynomial such that
      $\inner{p, q_x} = p(x)$.  The paper presents a way of finding this
      polynomial that is based on an orthonormal basis of the polynomial space.
      Instead of using that method, we use a different approach.  To find this
      polynomial $q_x$, first we compute $\inner{p,q_x}$:

      \begin{align*}
        \int_0^1 p(t)q_x(t) dt &= p(x) \\
        \int_0^1 (p_0 + p_1t + p_2t^2)(q_0 + q_1t + q_2t^2)dt &= p(x) \\
        \int_0^1 (p_0q_0 + (p_0q_1 + p_1q_0)t + (p_0q_2 + p_1q_1 + p_2q_0)t^2 + (p_1q_2 + p_2q_1)t^3 + p_2q_2t^4)dt &= p(x) \\
        p_0q_0 + \frac{1}{2}p_0q_1 + \frac{1}{2}p_1q_0 + \frac{1}{3}p_0q_2 + \frac{1}{3}p_1q_1 + \frac{1}{3}p_2q_0 + \frac{1}{4}p_1q_2 + \frac{1}{4}p_2q_1 + \frac{1}{5}p_2q_2 &= p(x) \\
      \end{align*}

      For each value of $x$, we have a system of three equations and three
      unknowns as a result.  We will show the system for $x=0$ without loss of
      generality.  Since $p(0) = p_0$, the above equation becomes:

      \begin{equation*}
        p_0q_0 + \frac{1}{2}p_0q_1 + \frac{1}{2}p_1q_0 + \frac{1}{3}p_0q_2 + \frac{1}{3}p_1q_1 + \frac{1}{3}p_2q_0 + \frac{1}{4}p_1q_2 + \frac{1}{4}p_2q_1 + \frac{1}{5}p_2q_2 = p_0
      \end{equation*}

      This can be represented as the following system of equations (one for the
      coefficients of each $p_i$):

      \begin{align*}
        q_0 + \frac{1}{2} q_1 + \frac{1}{3} q_2 &= 1 \\
        \frac{1}{2}q_0 + \frac{1}{3}q_1 + \frac{1}{4}q_2 &= 0 \\
        \frac{1}{3}q_0 + \frac{1}{4}q_1 + \frac{1}{5}q_2 &= 0 \\
      \end{align*}

      Using NumPy to solve the system, we obtain: $q_0 = 9$, $q_1 = -36$,
      $q_2 = 30$, which gives the polynomial $q_0(t) = 9 - 36t + 30t^2$.  The
      two remaining polynomials may be found similarly:
      $q_{\frac{1}{2}}(t) = -1.5 + 15 t - 15 t^2$, $q_1(t) = 3 - 24t + 30t^2$.

      Note that for each value of $x$, the only thing that changes is the right
      hand side of the equations.  So, this system could be represented in
      matrix form as $Aq = b$, where $b$ is the vector of coefficients of $p_i$
      on the right hand side of the original equation.  $b = (1, 0, 0)$ in the
      example of $x=0$, and in the general case, $b = (1, x, x^2)$.  The general
      solution, therefore, is $q = A^{-1}b$.  Via NumPy, we find:

      \begin{equation*}
        A^{-1} = \begin{bmatrix*}[r] 9 & -36 & 30 \\ -36 & 192 & -180 \\ 30 &-180 & 180 \end{bmatrix*}
      \end{equation*}

      Therefore, in the general case, we can compute $A^{-1}b$ and use those
      coefficients to come up with a general $q_x$:

      \begin{equation*}
        \phi(x) = q_x(t) = (9 - 36x + 30x^2) + (-36 + 192x - 180x^2)t + (30 -
        180x + 180x^2)t^2
      \end{equation*}

      Note that this is the same result found in the final page of the provided
      paper (see $k(x,y)$).  This provides some validation of this (rather
      interesting but also rather involved) method of determining the
      representers.

    \item Start with the original equation:

      \begin{align*}
        \inner{w, \phi(x_i)} &= y_i \\
        \inner{\sum_{j} \alpha_j \phi(x_j), \phi(x_i)} &= y_i \\
        \sum_j \alpha_j \inner{\phi(x_j), \phi(x_i)} &= y_i \\
        \sum_j \alpha_j K(x_j, x_i) &= y_i \\
      \end{align*}

    \item The kernel matrix:

      \begin{align*}
        K &=
        \begin{bmatrix*}
          (0 * 0 + 1)^2 & (0 * \sfrac{1}{2} + 1)^2 & (0 * 1 + 1)^2 \\
          (\sfrac{1}{2} * 0 + 1)^2 & (\sfrac{1}{2} * \sfrac{1}{2} + 1)^2 & (\sfrac{1}{2} * 1 + 1)^2 \\
          (1 * 0 + 1)^2 & (1 * \sfrac{1}{2} + 1)^2 & (1*1 + 1)^2 \\
        \end{bmatrix*} \\
        K &=
        \begin{bmatrix*}
          1 & 1 & 1 \\
          1 & \sfrac{25}{16} & \sfrac{9}{4} \\
          1 & \sfrac{9}{4} & 4 \\
        \end{bmatrix*} \\
      \end{align*}

      So, the system of equations from (b) becomes:

      \begin{align*}
        \alpha_1 + \alpha_2 + \alpha_3 &= 2 \\
        \alpha_1 + \frac{25}{16}\alpha_2 + \frac{9}{4} \alpha_3&= -1 \\
        \alpha_1 +\frac{9}{4}\alpha_2 + 4\alpha_3 &= 0 \\
      \end{align*}

      Solving via NumPy, we obtain $\alpha_1 = 33$, $\alpha_2 = -52$,
      $\alpha_3 = 21$.  Finally, we can compute the value of the polynomial
      corresponding to $w$:

      \begin{align*}
        w(t) &= \sum \alpha_i \phi(x_i) \\
        &= 33 q_0(t) - 52 q_{\frac{1}{2}}(t) + 21 q_1(t) \\
        &= 33 (9 - 36t + 30t^2) - 52 (-1.5 + 15 t - 15t^2) + 21 (3 - 24t + 30t^2) \\
        &= 297 - 1188t + 990t^2 + 78 - 780t + 780t^2 + 63 - 504t + 630t^2 \\
        &= 438 - 2472t + 2400t^2
      \end{align*}

      To be perfectly honest, I'm not sure what the connection is between this
      polynomial and the hyperplane for the feature space.
    \end{enumerate}
  \end{problem}

\end{document}