\documentclass[fleqn]{homework}

\student{Stephen Brennan (smb196)}
\course{EECS 440}
\assignment{Programming 1}
\duedate{September 24, 2015}

%\usepackage{mathtools}
%\usepackage{graphicx}

\begin{document}
  \maketitle

  \begin{problem}{a}
    \begin{question}
      What is the accuracy of the classifier on each dataset when the depth is
      set to 1? (i.e. the tree has just 1 test)
    \end{question}
    To answer this question, I set the depth to be 2, because my implementation
    includes the leaf nodes in the accounting of depth.

    \begin{tabular}{lll}
      \textbf{Dataset} & \textbf{Accuracy} & \textbf{Majority Prop.} \\
      \hline
      \textit{Voting} & 0.984 & 0.555 \\
      \textit{Volcanoes} & 0.674 & 0.672 \\
      \textit{Spam} & 0.643 & 0.625 \\
    \end{tabular}

    I've provided the accuracy at depth 2 along with the majority class label,
    since the baseline for accuracy comparison should be the accuracy you would
    get by predicting the majority class label, which is simply the proportion
    of examples which have the majority class label.  By showing this on the
    same chart, you can see that \textit{voting} was by far the most successful
    with just a single test, and that although \textit{spam} had a lower
    accuracy than \textit{volcanoes} with a single test, it represented a much
    more meaningful improvement from the baseline than \textit{volcanoes} had.
  \end{problem}

  \begin{problem}{b}
    \begin{question}
      For \textit{spam} and \textit{voting}, look at first test picked by your
      tree. Do you think this intuitively looks like a sensible test to perform
      for these problems?  Explain.
    \end{question}

    For \textit{voting}, the first test picked by my tree was attribute 0.
    Unfortunately, the \texttt{voting.info} file did not provide reference for
    what each bill is.  However, in the \textit{voting} problem, it would make
    sense that any partisan bill would be highly predictive.

    For \textit{spam}, the first test picked by my tree was ``is the
    geographical distance less than 2.69''.  If true, it would predict negative
    (not spam).  If false, it would predict positive (spam).  This makes a lot
    of sense, given that a person is probably more likely to want mail from
    somebody close to them, and spam is more likely sent from a central place
    that is far away from most of its recipients.  This strategy alone results
    in 64.3\% accuracy in 5-fold validation, if you limit to a depth of two.
  \end{problem}

  \begin{problem}{c}
    \begin{question}
      For \textit{volcanoes} and \textit{spam}, plot the accuracy as the depth
      of the tree is increased. On the x-axis, choose depth values to test so
      there are at least five evenly spaced points. Does the accuracy improve
      smoothly as the depth of the tree increases? Can you explain the pattern
      of the graph?
    \end{question}
  \end{problem}

\end{document}